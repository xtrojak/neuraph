\documentclass[12pt]{article}
\usepackage[slovak]{babel}
\usepackage[utf8x]{inputenc}
\usepackage{kpfonts}
\usepackage{hyperref}
\usepackage{amsthm}

\newcommand{\mysection}[1]{{\newpage\centering\Large\textbf{#1}\\}\normalsize\vspace{0.1cm}}
\newcommand{\mysmallsection}[1]{\vspace{0.5cm}{\centering\large\textbf{#1}\\}\normalsize\vspace{0.5cm}}

\begin{document}

\mysection{PV021 Neuronové siete \\ Záverečná správa}
\begin{center}
Adam Krupička, Matej Troják
\end{center}

\mysmallsection{Problém šachového ťahu}
V našom projekte sme chceli riešiť problém pohybu šachových figúrok po šachovnici. Napadlo nás niekoľko prístupov, ako tento problém riešiť:

\begin{enumerate}
\item dať sieti na vstup celú šachovnicu a na výstupe očakávať opäť šachovni-cu, avšak pozmenenú o jeden validný ťah,
\item dať sieti na vstup pozície jednotlivých figúrok a na výstupe očakávať pozmenené pozície figúrok,
\item dať sieti na vstup celú šachovnicu/pozície figúrok a očakávať na výstupe ťah, ktorý sa má vykonať, t.j. súradnice políčka s \textit{nejakou} figúrkou a súradnice, kam sa má posunúť,
\item dať sieti na vstup dve šachovnice a sieť odpovie hodnotou medzi 0 a 1 ako \textit{dobrý} je ťah.
\end{enumerate}

{\centering\textbf{Zdroj dát}\\}

Použili sme záznamy šachových partií z Games of World Champions\footnotemark[1]\footnotetext[1]{\url{http://www.chess.com/download/view/games-of-world-champions}}. Na dáta sme využili voľne dostupný parser, ktorý ich prekonvertoval do formátu FEN (Forsyth–Edwards Notation), čo je štandartná notácia pre popis priebehu šachovej partie tak, aby mohla byť opätovne zrekonštruo-vaná. \\

{\centering\textbf{Implementácia}\\}

Implementovali sme viacvrstvú sieť (multilayer perceptron) a zpätnú propagáciu (backpropagation algorithm) v jazyku C++. Implementovali sme triedy pre neurón, sieť a šachovnicu. Urobili sme serializáciu siete tak, aby sme po ukončení učenia s ňou mohli znova pracovať aj s odstupom času. Celý program má 3 rôzne funkcie:

\begin{itemize}
\item \textbf{new} - vytvorenie novej siete, parameter je \textit{"počet neurónov vo vrstvách"} a názov siete
\item \textbf{learn} - spustí učenie pomocou spätnej propagácie nad daným vstupným súborom dát a sieťou
\item \textbf{eval} - vyhodnotí sieť nad daným vstupom pre danú sieť
\end{itemize}

\textbf{TBA} spustenie, kompilacia, poziadavky na c++ ...

\mysmallsection{Prístup 1}

Vstup tvorený FENmi sme jednoduchým scriptom previedli do sekvencie 64 číslic v rozsahu hodnôt [-6, 6]. Figúrky majú nasledujúcu číselnú reprezentáciu (jedná sa o biele figúrky, čierne majú opačné znamienko):

\begin{center}
{\small
\hspace*{-1cm}\begin{tabular}{| l l | l l |}
\hline
 pešiak & 1 & strelec & 4 \\
 veža & 2 & kráľ & 5 \\
 kôň & 3 & kráľovná & 6\\
\hline
\end{tabular}
}
\end{center}

prázdne políčko má hodnotu 0.

{\centering\textbf{Riešenie}\\}

Nakoľko sme potrebovali rozsah hodnôt [-6, 6], ako aktivačnú funkciu sme zvolili hyperbolický tangens s upravenými koeficientami tak, aby bol na danom intervale približne lineárny. 

\begin{equation}
F_a(x)=12 \times tanh(\frac{x}{12})
\end{equation}

Trénovacie dáta boli dvojice sekvencií 64 číslic. Ťah je reprezentovaný tak, že na políčko, \textbf{z} ktorého sa figúrka hýbe, sa priradí číslo nula a políčko, \textbf{na} ktoré sa figúrka hýbe dostane pôvodnú hodnotu figúrky. Nakoľko v trénovacích dátach boli ťahy ako pre bieleho, tak i čierneho hráča, rozhodli sme sa používať iba biele ťahy. Tým sme dáta zmenšili o jenen rozmer (každý ťah čierneho hráča sa dá previesť na ťah bieleho hráča).

\textbf{TBA} nastavenia siete\\
\textbf{TBA} výsledky??

\mysmallsection{Prístup 2}

Vstup tvorený FENmi sme iným scriptom previedli do sekvencie 64 číslic v rozsahu hodnôt [0, 8]. Čísla sú vnímané ako dvojice, kde každá dvojica predstavuje súradnice konkrétnej figúrky na šachovnici. Poradie figúrok je teda fixne dané. Ak sa figúrka na šachovnici nenachádza, dostane súradnice (0, 0).

{\centering\textbf{Riešenie}\\}

Pre potrebný rozsah hodnôt sme zvolili funkciu:

\[ F_a(x) =
  \begin{cases}
    \quad 0  & \quad x \leq 0 \\
    \quad x  & \quad x > 0\\
  \end{cases}
\]

Ťah je reprezentovaný zmenou súradníc jednej figúrky. V prípade, ak sa jedná o ťah, kedy nepriateľ stráca figúrku (t.j. figúrka sa presúva na súradnice už zahrnuté v sekvencii súradníc), súradnice vyhodenej figúrky sú (0, 0).

\textbf{TBA} nastavenia siete\\
\textbf{TBA} výsledky??

\mysmallsection{Prístup 3}

Tento prístup je veľmi podobný prístupu číslo 2, rozdiel je iba vo výstupe siete. Namiesto toho, aby bol ťah zahrnutý v rozložení novej šachovnice, je výstupom explicitne daný ťah, t.j. súradnice políčka, \textbf{z} ktorého sa má figúrka pohnúť a súradnice \textbf{kam} sa má figúrka pohnúť.

\end{document}