\documentclass[12pt]{article}
\usepackage[slovak]{babel}
\usepackage[utf8x]{inputenc}
\usepackage{kpfonts}
\usepackage{hyperref}
\usepackage{amsthm}
\usepackage{tikz}

\newcommand{\mysection}[1]{{\newpage\centering\Large\textbf{#1}\\}\normalsize\vspace{0.1cm}}
\newcommand{\mysmallsection}[1]{\vspace{0.5cm}{\centering\large\textbf{#1}\\}\normalsize\vspace{0.5cm}}

\def\layersep{2.5cm}

\begin{document}

\mysection{PV021 Neuronové siete \\ Záverečná správa}
\begin{center}
Adam Krupička, Matej Troják
\end{center}

\mysmallsection{Problém šachového ťahu}
V našom projekte sme chceli riešiť problém pohybu šachových figúrok po šachovnici. Napadlo nás niekoľko prístupov, ako tento problém riešiť:

\begin{enumerate}
\item vstupom je celá šachovnica a výstupom opäť šachovnica, avšak pozmenená o jeden validný ťah,
\item vstupom sú pozície jednotlivých figúrok a výstupom pozmenené pozí-cie figúrok,
\item vstupom je celá šachovnica/pozície figúrok a výstupom ťah, ktorý sa má vykonať, t.j. súradnice políčka s \textit{nejakou} figúrkou a súradnice, kam sa má posunúť,
\item vstupom sú dve šachovnice a sieť odpovie hodnotou medzi 0~a~1 ako \textit{dobrý} je ťah (inšpirované iným zdrojom, neriešené)
\end{enumerate}

{\centering\textbf{Zdroj dát}\\}

Použili sme záznamy šachových partií z Games of World Champions\footnotemark[1]\footnotetext[1]{\url{http://www.chess.com/download/view/games-of-world-champions}}. Na dáta sme využili voľne dostupný parser, ktorý ich prekonvertoval do~for-mátu FEN (Forsyth–Edwards Notation), čo je štandardná notácia pre popis priebehu šachovej partie tak, aby mohla byť opätovne zrekonštruovaná. \\

{\centering\textbf{Implementácia}\\}

Implementovali sme viacvrstvú sieť (multilayer perceptron) a spätnú propagáciu (backpropagation algorithm) v jazyku C++. Implementovali sme triedy pre neurón, sieť a šachovnicu. Urobili sme serializáciu siete tak, aby sme po ukončení učenia s ňou mohli znova pracovať. Celý program má 3 rôzne funkcie:

\begin{itemize}
\item \textbf{new} - vytvorenie novej siete, parameter je \textit{"počet neurónov vo vrstvách"}

\hspace{1cm} a názov siete

{\centering\small ./main~new~`3~81~2'~my.net\\}

\item \textbf{learn} - spustí učenie pomocou spätnej propagácie nad daným vstup-

\hspace{1.3cm}ným súborom dát a sieťou

{\centering\small ./main learn input.txt my.net \\}

\item \textbf{eval} - vyhodnotí sieť nad daným vstupom pre danú sieť

{\centering\small ./main eval `8 5 1 5 8 4 1 4 8 3 8 6 1 3 ... 2 2 3 2 4 2 5 2 6 2 7 2 8' my.net \\}

\end{itemize}

\mysmallsection{Prístup 1}

Vstup tvorený FENmi sme jednoduchým skriptom previedli do sekvencie 64 číslic v rozsahu hodnôt [-6, 6]. Figúrky majú nasledujúcu číselnú reprezentáciu (jedná sa o biele figúrky, čierne majú opačné znamienko):

\begin{center}
{\small
\hspace*{-1cm}\begin{tabular}{| l l | l l |}
\hline
 pešiak & 1 & strelec & 4 \\
 veža & 2 & kráľ & 5 \\
 kôň & 3 & kráľovná & 6\\
\hline
\end{tabular}
}
\end{center}

\noindent a prázdne políčko má hodnotu 0.

{\vspace*{0.5cm}\centering\textbf{Riešenie}\\}

Nakoľko sme potrebovali rozsah hodnôt [-6, 6], ako aktivačnú funkciu sme zvolili hyperbolický tangens s upravenými koeficientmi tak, aby bol na danom intervale približne lineárny. 

\begin{equation}
F_a(x)=12 \times tanh(\frac{x}{12})
\end{equation}

Trénovacie dáta boli dvojice sekvencií 64 číslic. Ťah je reprezentovaný tak, že na políčko, \textbf{z} ktorého sa figúrka hýbe, sa priradí číslo nula a políčko, \textbf{na} ktoré sa figúrka hýbe, dostane pôvodnú hodnotu figúrky. Nakoľko v~tré-novacích dátach boli ťahy ako pre bieleho, tak i čierneho hráča, rozhodli sme sa používať iba biele ťahy. Tým sme dáta zmenšili o jeden rozmer (každý ťah čierneho hráča sa dá previesť na ťah bieleho hráča).

\newpage

\mysmallsection{Prístup 2}

Vstup tvorený FENmi sme iným skriptom previedli do sekvencie 64 číslic v rozsahu hodnôt [0, 8]. Čísla sú vnímané ako dvojice, kde každá dvojica predstavuje súradnice konkrétnej figúrky na šachovnici. Poradie figúrok je teda fixne dané. Ak sa figúrka na šachovnici nenachádza, dostane súradnice (0, 0).

{\vspace*{0.5cm}\centering\textbf{Riešenie}\\}

Pre potrebný rozsah hodnôt sme zvolili funkciu:

\[ F_a(x) =
  \begin{cases}
    \quad 0  & \quad x \leq 0 \\
    \quad x  & \quad x > 0\\
  \end{cases}
\]

Ťah je reprezentovaný zmenou súradníc jednej figúrky. V prípade, ak sa jedná o ťah, kedy súper stráca figúrku (t.j. figúrka sa presúva na~súrad-nice už zahrnuté v sekvencii súradníc), súradnice vyhodenej figúrky sú nastavené na hodnoty (0,~0).

\mysmallsection{Prístup 3}

Tento prístup je veľmi podobný prístupu číslo 2, rozdiel je iba vo výstu-pe siete. Namiesto toho, aby bol ťah zahrnutý v rozložení novej šachovnice, je výstupom explicitne daný ťah, t.j. súradnice políčka, \textbf{z} ktorého sa má figúrka pohnúť a súradnice \textbf{kam} sa má figúrka pohnúť.

\mysmallsection{Výsledky}

Problémom u všetkých typov prístupov bolo nájsť parametre siete tak, aby sa daný problém naučila vyhodnocovať. Skúšali sme mnoho kombinácii topológie siete a rýchlosti učenia, no vždy neúspešne. Pri prístupe 3, kedy funkčnosť siete vyzerala nádejne (nastavenie: 64, 128, 64, 16, 4), bol zase problém z výpočetného hľadiska. Odhad učenia na našich dátach bol asi tri dni a nemali sme prostriedky ako otestovať, či je sieť schopná sa tomuto prístupu naučiť.

Preto sme rozhodli riešiť pomocou neurónových sieti iný problém.

\newpage
\mysmallsection{Problém derivácie polynómu}

\noindent Ideou je, že sa sieti na vstup dá polynóm a ona vráti jeho deriváciu. 

{\vspace*{0.5cm}\centering\textbf{Príklad}\\}

\noindent Majme vstupný polynóm:

\begin{equation}
f(x)=5x^3 + 2x + 8
\end{equation}

Takýto polynóm bude interpretovaný pre sieť nasledovne (uvažujme, že sieť je schopná načítavať polynómy stupňa maximálne 5):

\begin{equation}
f(x)=8x^0 + 2x^1 + 0x^2 + 5x^3 + 0x^4 + 0x^5
\end{equation}

\noindent takže finálnym vstupom pre sieť je sekvencia čísel "\textbf{8 2 0 5 0 0}".

Chceme dostať deriváciu tohto polynómu

\begin{equation}
f'(x)=2 + 15x^2
\end{equation}

\noindent čiže sekvenciu čísel "\textbf{2 0 15 0 0}", ktorá je o jednu číslicu kratšia.

{\vspace*{0.5cm}\centering\textbf{Zdroj dát}\\}

Ako zdroj dát sme zvolili funkciu, ktorá vygeneruje náhodný polynóm z daného rozsahu hodnôt (maximálny stupeň polynómu a rozsah koeficientov) a zároveň vypočíta jeho deriváciu. 

{\vspace*{0.5cm}\centering\textbf{Implementácia}\\}

Implementácia čo sa siete týka sa nezmenila. Pribudla trieda Polynoms, ktorá pracuje s polynómami. Zásadnou je funkcia, ktorá generuje dáta pre sieť, t.j. dvojice vstup a výstup. Pre rozhranie to má dopad na~nasledujúcu funkcionalitu:

\begin{itemize}
\item \textbf{learn} - spustí učenie pomocou spätnej propagácie nad daným

\hspace{1.2cm} počtom vstupných polynómov a sieťou

{\centering\small ./main learn 10000 my.net \\}

\item \textbf{eval} - vyhodnotí sieť nad daným vstupom pre danú sieť

{\centering\small ./main eval `2 1 3' my.net \\}

\end{itemize}

{\vspace*{0.5cm}\centering\textbf{Riešenie}\\}

Nakoľko rozsah koeficientov, ktoré môžu byť v polynóme použité je z všeobecného hľadiska neobmedzený, rozhodli sme sa použiť lineárnu funkciu: 

\begin{equation}
F_a(x)=x~,
\end{equation}

\noindent ktorá má v každom bode deriváciu rovnú 1.  

\mysmallsection{Výsledky}

Prvým pokusom bolo realizovať derivovanie pomocou neurónovej siete na veľmi obmedzených polynómoch. Zvolili sme maximálny stupeň polynómu 3 a rozsah koeficientov ako [0, 3]. Topológiu siete sme zvolili "3~81~2". Problém bol v rýchlosti učenia, ktorú sme museli nastaviť na~veľmi malú hodnotu, aby sa sieť správne učila -- 0.0001.

Následne po dostatočnom učení (asi 20000 náhodne vygenerovaných polynómov) sme dosiahli pozitívne výsledky vo výstupoch siete. Tu je niekoľko príkladov:

\begin{itemize}
\item \begin{tabular}{ l l }
		vstup: & $2 + 1x^1 + 3x^2$\\
		derivácia: & $1.05352 + 5.89288x^1 $
	  \end{tabular}
\item \begin{tabular}{ l l }
		vstup: & $2 + 1x^1$\\
		derivácia: & $1.00556 + 0.129582x^1$
	  \end{tabular}
\item \begin{tabular}{ l l }
		vstup: & $3 + 3x^1 + 3x^2$\\
		derivácia: & $2.93755 + 5.82714x^1$
	  \end{tabular}
\end{itemize}

Druhým výsledkom je iný prístup k učeniu. Namiesto toho, aby sme sieť učili konkrétne polynómy, dávali sme jej na vstup vždy iba jeden člen s nenulovým koeficientom. Tým sme chceli dosiahnuť \textit{nezávislosť} jednotlivých členov medzi sebou. Zvolili sme maximálny stupeň polynómu 4 a~rozsah koeficientov ako [0, 10]. To znamená, že dokopy sme sieť učili iba sadou dát, kde sa vyskytovalo dokopy 55 rôznych polynómov. Zvolili sme minimálnu topológiu siete, čiže "5~4". Ideálna sieť by vyzerala takto:

\begin{center}
\begin{tikzpicture}[shorten >=1pt,->,draw=black!50, node distance=\layersep]
    \tikzstyle{every pin edge}=[<-,shorten <=1pt]
    \tikzstyle{neuron}=[circle,fill=black!25,minimum size=17pt,inner sep=0pt]
    \tikzstyle{input neuron}=[neuron, fill=green!50];
    \tikzstyle{output neuron}=[neuron, fill=red!50];
    \tikzstyle{annot} = [text width=4em, text centered]

    \foreach \name / \y in {1,...,5}
        \node[input neuron] (I-\name) at (-\y, 0) {};

	\foreach \name / \y in {1,...,4}
        \node[output neuron] (H-\name) at (-\y, \layersep) {};

    \foreach \source in {1,...,5}
        \foreach \dest in {1,...,4}
            \path (I-\source) edge (H-\dest);
    \path (I-1) [red] edge node [red, right] {$4$} (H-1);
    \path (I-2) [red] edge node [red, right] {$3$} (H-2);
    \path (I-3) [red] edge node [red, right] {$2$} (H-3);
    \path (I-4) [red] edge node [red, right] {$1$} (H-4);
     
    \node[annot,left of=I-5, node distance=2cm] (hl) {Input layer};
    \node[annot,above of=hl] {Output layer};
\end{tikzpicture}
\end{center}

\noindent kde ostatné hrany majú váhu 0. 

Preto sme sa pokúsili tento ideál dosiahnuť. Naša sieť si po učiacom procese nastavila na hranách takmer identické váhy, preto aj vracala celkom presné výsledky. Vďaka tomu sa naša sieť naučila pracovať s ľubovoľ-nými koeficientmi bez toho, aby sme ju to museli explicitne učiť. Nemá problém ani s desatinnými a zápornými číslami. Príklady:

\begin{itemize}
\item \begin{tabular}{ l l }
		vstup: & $20.5x^1 + 11.3x^2 + 2.5x^3 + -0.5x^4$\\
		derivácia: & $20.485 + 22.5747x^1 + 7.46458x^2 + -2.04699x^3$
	  \end{tabular}
\item \begin{tabular}{ l l }
		vstup: & $20.5x^1 + 11.3x^2 + 20x^3 + -500x^4$\\
		derivácia: & $20.7582 + 23.0364x^1 + 60.6112x^2 + -1999.19x^3$
	  \end{tabular}
\end{itemize}

\mysmallsection{Rozdelenie práce}

\begin{itemize}
\item Implementácia neurónovej siete $\rightarrow$ Adam Krupička

\item Spracovanie vstupu, záverečná správa $\rightarrow$ Matej Troják

\item Kreatívna činnosť a iné $\rightarrow$ spoločne
\end{itemize}

\end{document}