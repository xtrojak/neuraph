\documentclass[12pt]{article}
\usepackage[slovak]{babel}
\usepackage[utf8x]{inputenc}
\usepackage{kpfonts}
\usepackage{hyperref}
\usepackage{amsthm}

\newcommand{\mysection}[1]{{\newpage\centering\Large\textbf{#1}\\}\normalsize\vspace{0.1cm}}
\newcommand{\mysmallsection}[1]{\vspace{0.5cm}{\centering\large\textbf{#1}\\}\normalsize\vspace{0.5cm}}

\begin{document}

\mysection{PV021 Neuronové siete \\ Záverečná správa}
\begin{center}
Adam Krupička, Matej Troják
\end{center}

\mysmallsection{Problém šachového ťahu}
V našom projekte sme chceli riešiť problém pohybu šachových figúrok po šachovnici. Napadlo nás niekoľko prístupov, ako tento problém riešiť:

\begin{enumerate}
\item dať sieti na vstup celú šachovnicu a na výstupe očakávať opäť šachovnicu, avšak pozmenenú o jeden validný ťah,
\item dať sieti na vstup pozície jednotlivých figúrok a na výstupe očakávať pozmenené pozície figúrok,
\item dať sieti na vstup celú šachovnicu/pozície figúrok a očakávať na výstupe ťah, ktorý sa má vykonať, t.j. súradnice políčka s \textit{nejakou} figúrkou a súradnice, kam sa má posunúť,
\item dať sieti na vstup dve šachovnice a sieť odpovie hodnotou medzi 0 a 1 ako \textit{dobrý} je ťah.
\end{enumerate}

{\centering\textbf{Zdroj dát}\\}

Použili sme záznamy šachových partií z Games of World Champions\footnotemark[1]\footnotetext[1]{\url{http://www.chess.com/download/view/games-of-world-champions}}. Na dáta sme využili voľne dostupný parser, ktorý ich prekonvertoval do formátu FEN (Forsyth–Edwards Notation), čo je štandartná notácia pre popis priebehu šachovej partie tak, aby mohla byť opätovne zrekonštruovaná. 

{\centering\textbf{Implementácia}\\}

Implementovali sme viacvrstvú sieť (multilayer perceptron) a zpätnú propagáciu (backpropagation algorithm) v jazyku C++.

\textbf{TBA} rozhranie

\newpage
\mysmallsection{Prístup 1}

Vstup tvorený FENmi sme jednoduchým scriptom previedli do sekvencie 64 číslic v rozsahu hodnôt [-6, 6]. Figúrky majú nasledujúcu číselnú reprezentáciu (jedná sa o biele figúrky, čierne majú opačné znamienko):

\begin{center}
{\small
\hspace*{-1cm}\begin{tabular}{| l l | l l |}
\hline
 pešiak & 1 & strelec & 4 \\
 veža & 2 & kráľ & 5 \\
 kôň & 3 & kráľovná & 6\\
\hline
\end{tabular}
}
\end{center}

prázdne políčko má hodnotu 0.

{\centering\textbf{Riešenie}\\}

Nakoľko sme potrebovali rozsah hodnôt [-6, 6], ako aktivačnú funkciu sme zvolili hyperbolický tangens s upravenými koeficientami tak, aby bol na danom intervale približne lineárny. 

\begin{equation}
F_a=12 \times tanh(\frac{x}{12})
\end{equation}

\textbf{TBA} trénovacie dáta\\
\textbf{TBA} nastavenia siete\\
\textbf{TBA} výsledky??

\end{document}